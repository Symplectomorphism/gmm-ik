\section{Numerical Example}
\label{sec:numerical}
%

A Julia implementation of this algorithm may be found
in~\cite{satici_implementation}. The implementation allows the user to specify
how many data points to randomly generate and how many components to use in the
Gaussian mixture model. Once these are specified, the model can be trained by 
executing the \texttt{execute\_em!} function. 

The trained model may be tested by running the \texttt{test\_training} function,
which generates $200$ extra points and returns the average error incurred by the
output of the algorithm. A visualization of the performance of the algorithm is
depicted in Figure~\ref{fig:example_sol}.
